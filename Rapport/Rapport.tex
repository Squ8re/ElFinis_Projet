\documentclass{article}
\usepackage[utf8]{inputenc}
\usepackage[siunitx]{circuitikz}
\usepackage{lmodern}

\usepackage{amsmath}
\usepackage{soul}
\usepackage{amssymb}

\usepackage{mathrsfs}
\usepackage{siunitx}

\usepackage{lettrine}
\usepackage{graphicx}%permet de mettre des images, les réduire, les placer, les tourner [scale=....,angle=...]
\usepackage{pdfpages}%permet d'inclure des pages de documents pdfs
\usepackage{booktabs}%permet de faire des jolis tableaux
\usepackage{multicol}%permet de faire des listes sur 2 colonnes
\usepackage{multirow}
\usepackage{geometry}%changer les marges, footers, headers
\usepackage{float} % permet de choisir mieux ou mettre les images avec[H],t,b,p,
\usepackage[T1]{fontenc}
\usepackage{enumitem}
\usepackage{xcolor}%mettre du txt en couleur \textcolor{colorname}{Text to be colored}
\usepackage{textcomp,gensymb} %permet le \degree \perthousand \micro
\usepackage{amsmath}%faire des équas de math alignées et des théorèmes
\usepackage[french]{babel}%met le document en francais (table de matières
\usepackage[T1]{fontenc}%pour avoir les accents et les caractères bizarres
\usepackage{rotating}% permet de tourner l'image avec \begin{sidewaysfigure}
\usepackage{fancyhdr}%modifier les headers et footers
\usepackage{hyperref}%permet de cliquer sur les urls et les sections dans les pdfs
\usepackage{titling}
\usepackage{wrapfig}
\usepackage{listingsutf8}  % Insertion de code
\begin{document}
% Template issu de: https://www.latextemplates.com/template/academic-title-page
\begin{titlepage} % Suppresses displaying the page number on the title page and the subsequent page counts as page 1
	\newcommand{\HRule}{\rule{\linewidth}{0.5mm}} % Defines a new command for horizontal lines, change thickness here
	
	%------------------------------------------------
	%   Logos
	%------------------------------------------------
	\begin{minipage}[t]{0.30\linewidth}
		\includegraphics[height=1.5cm]{Images/logo_EPL.jpg}
	\end{minipage} \hfill
	\begin{minipage}[t]{0.35\linewidth}
		\includegraphics[height=1.5cm]{Images/logo_UCL.jpg}
	\end{minipage}
	
	\center % Centre everything on the page
	
	
	%------------------------------------------------
	%	Headings
	%------------------------------------------------
	
	%\textsc{\LARGE Ecole Polytechnique de Louvain}\\[1.5cm] % Main heading such as the name of your university/college
	
	\vspace{1.5cm}
	\textsc{\Large Introduction aux éléments finis}\\[0.5cm] % Major heading such as course name
	
	\textsc{\large MECA1120}\\[1.0cm] % Minor heading such as course title
	
	%------------------------------------------------
	%	Title
	%------------------------------------------------
	
	\HRule\\[0.65cm]
	%\vspace{1.5cm}
	
	{\huge\bfseries Projet 18-19 : note de synthèse}\\[0.4cm] % Title of your document
	
	\HRule\\[1.5cm]
	%\vspace{1.5cm}
	
	%------------------------------------------------
	%	Author(s)
	%------------------------------------------------
	
	%{\large\textit{Auteurs}}\\
	Groupe n\degree$120$\\[0.2cm]
	Amadéo \textsc{David}  - $4476 1700$% Your name
	
	Nicolas \textsc{Bouchat} - $3488 1700$% Your name
	
	%------------------------------------------------
	%	Date
	%------------------------------------------------
	
	\vfill\vfill\vfill % Position the date 3/4 down the remaining page
	
	{\large 6 mai 2019} % Date, change the \today to a set date if you want to be precise
	
	%----------------------------------------------------------------------------------------
	
	\vfill % Push the date up 1/4 of the remaining page
	
\end{titlepage}
\section{Obtention des équations en eaux peu profondes}
	%TODO
\section{Ordre de précision du résultat obtenu}
	%TODO
\section{Analyse de la solution}
	%TODO
\section{Optimisation du programme}
	%TODO
\end{document}
