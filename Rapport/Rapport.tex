% !TeX spellcheck = fr_FR
\documentclass{article}
\usepackage[utf8]{inputenc}
\usepackage[siunitx]{circuitikz}
\usepackage{lmodern}

\usepackage{amsmath}
\usepackage{soul}
\usepackage{amssymb}

\usepackage{mathrsfs}
\usepackage{siunitx}

\usepackage{lettrine}
\usepackage{graphicx}%permet de mettre des images, les réduire, les placer, les tourner [scale=....,angle=...]
\usepackage{pdfpages}%permet d'inclure des pages de documents pdfs
\usepackage{booktabs}%permet de faire des jolis tableaux
\usepackage{multicol}%permet de faire des listes sur 2 colonnes
\usepackage{multirow}
\usepackage{geometry}%changer les marges, footers, headers
\usepackage{float} % permet de choisir mieux ou mettre les images avec[H],t,b,p,
\usepackage[T1]{fontenc}
\usepackage{enumitem}
\usepackage{xcolor}%mettre du txt en couleur \textcolor{colorname}{Text to be colored}
\usepackage{textcomp,gensymb} %permet le \degree \perthousand \micro
\usepackage{amsmath}%faire des équas de math alignées et des théorèmes
\usepackage[french]{babel}%met le document en francais (table de matières)
\usepackage[T1]{fontenc}%pour avoir les accents et les caractères bizarres
\usepackage{rotating}% permet de tourner l'image avec \begin{sidewaysfigure}
\usepackage{fancyhdr}%modifier les headers et footers
\usepackage{hyperref}%permet de cliquer sur les urls et les sections dans les pdfs
\usepackage{titling}
\usepackage{wrapfig}
\usepackage{listingsutf8}  % Insertion de code
\begin{document}
% Template issu de: https://www.latextemplates.com/template/academic-title-page
\begin{titlepage} % Suppresses displaying the page number on the title page and the subsequent page counts as page 1
	\newcommand{\HRule}{\rule{\linewidth}{0.5mm}} % Defines a new command for horizontal lines, change thickness here
	
	%------------------------------------------------
	%   Logos
	%------------------------------------------------
	\begin{minipage}[t]{0.30\linewidth}
		\includegraphics[height=1.5cm]{Images/logo_EPL.jpg}
	\end{minipage} \hfill
	\begin{minipage}[t]{0.35\linewidth}
		\includegraphics[height=1.5cm]{Images/logo_UCL.jpg}
	\end{minipage}
	
	\center % Centre everything on the page
	
	
	%------------------------------------------------
	%	Headings
	%------------------------------------------------
	
	%\textsc{\LARGE Ecole Polytechnique de Louvain}\\[1.5cm] % Main heading such as the name of your university/college
	
	\vspace{1.5cm}
	\textsc{\Large Introduction aux éléments finis}\\[0.5cm] % Major heading such as course name
	
	\textsc{\large MECA1120}\\[1.0cm] % Minor heading such as course title
	
	%------------------------------------------------
	%	Title
	%------------------------------------------------
	
	\HRule\\[0.65cm]
	%\vspace{1.5cm}
	
	{\huge\bfseries Projet 18-19 : notes de synthèse}\\[0.4cm] % Title of your document
	
	\HRule\\[1.5cm]
	%\vspace{1.5cm}
	
	%------------------------------------------------
	%	Author(s)
	%------------------------------------------------
	
	%{\large\textit{Auteurs}}\\
	Groupe n\degree$120$\\[0.2cm]
	Amadéo \textsc{David}  - $4476 1700$% Your name
	
	Nicolas \textsc{Bouchat} - $3488 1700$% Your name
	
	%------------------------------------------------
	%	Date
	%------------------------------------------------
	
	\vfill\vfill\vfill % Position the date 3/4 down the remaining page
	
	{\large 6 mai 2019} % Date, change the \today to a set date if you want to be precise
	
	%----------------------------------------------------------------------------------------
	
	\vfill % Push the date up 1/4 of the remaining page
	
\end{titlepage}
\section{Obtention des équations en eaux peu profondes}
	%TODO
\section{Ordre de précision du résultat obtenu}
	%TODO
	CM8 slide 28 -> erreur $ = h^{p+1}$
\section{Analyse de la solution}
	L'analyse de notre solution est faites de plusieurs observations importantes.
	La première est que le choix de l'intervalle temporel entre chaque itération est important.
	
	D'une part, si celui-ci est top petit le temps de calcul sera immense.
	Or, comme ce n'est pas les détails qui nous intéresse mais bien le déplacement global de la vague, un grand temps de calcul pour un petit déplacement de l'océan ne s'avère pas utile.
	De plus, comme la solution comporte déjà des imprécisions (comme l'utilisation d'un facteur proportionnel à la vitesse représentant l'ensemble des effets dissipatifs), les détails seront certainement faussés, l'erreur étant trop grande que pour être ignorée.
	
	D'autre part, si le pas de temps est trop grand, la fonction sort de sa zone de stabilité.
	La solution devient alors inutilisable et ne fait plus sens.
	Cela s'observe facilement sur l'illustration suivante.
	%TODO: Ajouter une méthode instable.
\section{Optimisation du programme}
	%TODO: Vérifier les étapes de l'optimisation algorithmique.
	L'optimisation d'un programme peut être divisée en deux parties. 
	La première est l'optimisation algorithmique et la seconde est l'optimisation du code. 
	Nous allons d'abord nous concentrer sur la première avant d'analyser la seconde.
	\subsection{Optimisation algorithmique}
		L'optimisation algorithmique a pour but de diminuer au maximum la complexité mathématique du système de résolution utilisé dans le programme.
		En pratique, cela se traduira par le choix de la méthode mathématique de résolution.
		Après décomposition du problème en différents sous-problèmes plus simples, il apparait que ces derniers sont proches de la matière vue au cours du quadrimestre, notamment lors des devoirs à rendre.
		Voici les différentes étapes de résolution pour lesquels nous avons opté.
		\begin{enumerate}
			\item 
				Notre algorithme commence par déterminer les fonctions de forme d'un élément ainsi que son jacobien.
			\item 
				On applique ensuite une méthode la méthode de Hammer pour l'intégration des triangles et une méthode de Gauss-Legendre pour les arêtes. 
				On interpole en fait les valeurs des vecteurs modèles afin de passer par les points d'intégration (ces-derniers sont donnés).
			\item 
				Commence ensuite le calcul des différents termes à ajouter.
				Ceux-ci sont donnés à la troisième page de l'énoncé sous forme d'équation.
			\item
				Sur les bords, les termes d'intégration vont être ajoutés aux vecteurs $E_i$, $U_i$ et $V_i$.
			\item 
				On résout ensuite en multipliant chaque vecteur par son coefficient inverse puis en le divisant pas le jacobien.
				Cette méthode est appelée un solveur de Riemann.
			\item 
				L'algorithme se termine en appliquant Euler explicite.
		\end{enumerate}
	\subsection{Optimisation du code}
		Cette sous-partie est elle-même composée de différents domaines.
		Un code python peut en effet être optimisé de plusieurs manières.
		Il est possible d'utiliser des paquets compilés contenant des fonctions déjà codées et beaucoup plus efficaces que celle que nous pourrions écrire.
		Nous avons utilisé le paquet "numpy" qui est spécialisé dans la manipulation de matrices, vecteurs et polynômes.
		Nous n'avons cependant pas souhaité utiliser d'autres paquets supplémentaires car nous avons estimé que notre maitrise de ces derniers n'était pas suffisante.
		
		La partie la plus importante de l'optimisation du code réside dans la suppression des boucles. 
		C'est principalement des modifications de ce type que nous avons faites.
		La plupart des boucles présentes dans notre code ont été supprimées et remplacée par une vectorisation des éléments.
		La majorité des calculs itératifs ont été remplacés par des calculs matriciels beaucoup plus rapide.
		La seconde optimisation a été de sortir des boucles tous les éléments ne nécessitant pas d'être redéfinis à chaque itération.
		La dernière optimisation que nous avons mise en place a été de supprimer tous les calculs intermédiaires que nous avons alors fait sur papier pour les remplacer par des versions beaucoup plus condensées et simplifiées des calculs précédemment utilisés.
		Une optimisation minime mais à ne pas oublier a été de remplacer l'ensemble des éléments élevés au carré par une simple multiplication (par exemple en remplaçant $x^2$ par $x*x$).
		Nous avons en effet remarqué que python gérait plus rapidement ce second calcul que le premier.
		La différence est minime mais est tout de même observable sur un code conséquent.
\end{document}
